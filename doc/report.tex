\documentclass[mask,]{apa6}

\usepackage{amssymb,amsmath}
\usepackage{ifxetex,ifluatex}
\usepackage{fixltx2e} % provides \textsubscript
\ifnum 0\ifxetex 1\fi\ifluatex 1\fi=0 % if pdftex
  \usepackage[T1]{fontenc}
  \usepackage[utf8]{inputenc}
\else % if luatex or xelatex
  \ifxetex
    \usepackage{mathspec}
    \usepackage{xltxtra,xunicode}
  \else
    \usepackage{fontspec}
  \fi
  \defaultfontfeatures{Mapping=tex-text,Scale=MatchLowercase}
  \newcommand{\euro}{€}
\fi
% use upquote if available, for straight quotes in verbatim environments
\IfFileExists{upquote.sty}{\usepackage{upquote}}{}
% use microtype if available
\IfFileExists{microtype.sty}{\usepackage{microtype}}{}

% Table formatting
\usepackage{longtable, booktabs}
\usepackage{lscape}
% \usepackage[counterclockwise]{rotating}   % Landscape page setup for large tables
\usepackage{multirow}		% Table styling
\usepackage{tabularx}		% Control Column width
\usepackage[flushleft]{threeparttable}	% Allows for three part tables with a specified notes section
\usepackage{threeparttablex}            % Lets threeparttable work with longtable

% Create new environments so endfloat can handle them
% \newenvironment{ltable}
%   {\begin{landscape}\begin{center}\begin{threeparttable}}
%   {\end{threeparttable}\end{center}\end{landscape}}

\newenvironment{lltable}
  {\begin{landscape}\begin{center}\begin{ThreePartTable}}
  {\end{ThreePartTable}\end{center}\end{landscape}}

  \usepackage{ifthen} % Only add declarations when endfloat package is loaded
  \ifthenelse{\equal{\string }{\string man}}{%
   \DeclareDelayedFloatFlavor{ThreePartTable}{table} % Make endfloat play with longtable
   % \DeclareDelayedFloatFlavor{ltable}{table} % Make endfloat play with lscape
   \DeclareDelayedFloatFlavor{lltable}{table} % Make endfloat play with lscape & longtable
  }{}%



% The following enables adjusting longtable caption width to table width
% Solution found at http://golatex.de/longtable-mit-caption-so-breit-wie-die-tabelle-t15767.html
\makeatletter
\newcommand\LastLTentrywidth{1em}
\newlength\longtablewidth
\setlength{\longtablewidth}{1in}
\newcommand\getlongtablewidth{%
 \begingroup
  \ifcsname LT@\roman{LT@tables}\endcsname
  \global\longtablewidth=0pt
  \renewcommand\LT@entry[2]{\global\advance\longtablewidth by ##2\relax\gdef\LastLTentrywidth{##2}}%
  \@nameuse{LT@\roman{LT@tables}}%
  \fi
\endgroup}


  \usepackage{graphicx}
  \makeatletter
  \def\maxwidth{\ifdim\Gin@nat@width>\linewidth\linewidth\else\Gin@nat@width\fi}
  \def\maxheight{\ifdim\Gin@nat@height>\textheight\textheight\else\Gin@nat@height\fi}
  \makeatother
  % Scale images if necessary, so that they will not overflow the page
  % margins by default, and it is still possible to overwrite the defaults
  % using explicit options in \includegraphics[width, height, ...]{}
  \setkeys{Gin}{width=\maxwidth,height=\maxheight,keepaspectratio}
\ifxetex
  \usepackage[setpagesize=false, % page size defined by xetex
              unicode=false, % unicode breaks when used with xetex
              xetex]{hyperref}
\else
  \usepackage[unicode=true]{hyperref}
\fi
\hypersetup{breaklinks=true,
            pdfauthor={},
            pdftitle={The title},
            colorlinks=true,
            citecolor=blue,
            urlcolor=blue,
            linkcolor=black,
            pdfborder={0 0 0}}
\urlstyle{same}  % don't use monospace font for urls

\setlength{\parindent}{0pt}
%\setlength{\parskip}{0pt plus 0pt minus 0pt}

\setlength{\emergencystretch}{3em}  % prevent overfull lines


% Manuscript styling
\captionsetup{font=singlespacing,justification=justified}
\usepackage{csquotes}
\usepackage{upgreek}



\usepackage{tikz} % Variable definition to generate author note

% fix for \tightlist problem in pandoc 1.14
\providecommand{\tightlist}{%
  \setlength{\itemsep}{0pt}\setlength{\parskip}{0pt}}

% Essential manuscript parts
  \title{The title}

  \shorttitle{Title}


  \author{First Author\textsuperscript{1}~\& Your Name\textsuperscript{1,2}}

  % \def\affdep{{"", ""}}%
  % \def\affcity{{"", ""}}%

  \affiliation{
    \vspace{0.5cm}
          \textsuperscript{1} Your institution\\
          \textsuperscript{2} Your co-author's institution  }

  \authornote{
    Correspondence concerning this article should be addressed to First
    Author, Postal address. E-mail:
    \href{mailto:your@email.com}{\nolinkurl{your@email.com}}
  }


  \abstract{One or two sentences providing a \textbf{basic introduction} to the
field, comprehensible to a scientist in any discipline.

Two to three sentences of \textbf{more detailed background},
comprehensible to scientists in related disciplines.

One sentence clearly stating the \textbf{general problem} being
addressed by this particular study.

One sentence summarizing the main result (with the words ``\textbf{here
we show}'' or their equivalent).

Two or three sentences explaining what the \textbf{main result} reveals
in direct comparison to what was thought to be the case previously, or
how the main result adds to previous knowledge.

One or two sentences to put the results into a more \textbf{general
context}.

Two or three sentences to provide a \textbf{broader perspective},
readily comprehensible to a scientist in any discipline.}
  \keywords{keywords \\

    \indent Word count: X
  }

\usepackage[titles]{tocloft}
\cftpagenumbersoff{figure}
\renewcommand{\cftfigpresnum}{\itshape\figurename\enspace}
\renewcommand{\cftfigaftersnum}{.\space}
\setlength{\cftfigindent}{0pt}
\setlength{\cftafterloftitleskip}{0pt}
\settowidth{\cftfignumwidth}{Figure 10.\qquad}

\cftpagenumbersoff{table}
\renewcommand{\cfttabpresnum}{\itshape\tablename\enspace}
\renewcommand{\cfttabaftersnum}{.\space}
\setlength{\cfttabindent}{0pt}
\setlength{\cftafterloftitleskip}{0pt}
\settowidth{\cfttabnumwidth}{Table 10.\qquad}

\usepackage{endnotes}
\let\footnote\endnote


\usepackage{amsthm}
\newtheorem{theorem}{Theorem}[section]
\newtheorem{lemma}{Lemma}[section]
\theoremstyle{definition}
\newtheorem{definition}{Definition}[section]
\newtheorem{corollary}{Corollary}[section]
\newtheorem{proposition}{Proposition}[section]
\theoremstyle{definition}
\newtheorem{example}{Example}[section]
\theoremstyle{definition}
\newtheorem{exercise}{Exercise}[section]
\theoremstyle{remark}
\newtheorem*{remark}{Remark}
\newtheorem*{solution}{Solution}
\begin{document}

\maketitle

\setcounter{secnumdepth}{0}



\hypertarget{methods}{%
\section{Methods}\label{methods}}

We report how we determined our sample size, all data exclusions (if
any), all manipulations, and all measures in the study.

\hypertarget{participants}{%
\subsection{Participants}\label{participants}}

\hypertarget{material}{%
\subsection{Material}\label{material}}

\hypertarget{procedure}{%
\subsection{Procedure}\label{procedure}}

\hypertarget{data-analysis}{%
\subsection{Data analysis}\label{data-analysis}}

We used R (Version 3.4.4; R Core Team, 2018) and the R-packages
\emph{dplyr} (Version 0.7.6; Wickham et al., 2018), \emph{ggalt}
(Version 0.4.0; Rudis, Bolker, \& Schulz, 2017), \emph{ggplot2} (Version
3.0.0.9000; Wickham, Chang, et al., 2018), \emph{ggthemes} (Version
3.4.0; Arnold, 2018), \emph{here} (Version 0.1; Müller, 2017),
\emph{knitr} (Version 1.20; Xie, 2018a), \emph{papaja} (Version
0.1.0.9709; Aust \& Barth, 2018), \emph{patchwork} (Version 0.0.1;
Pedersen, 2017), \emph{readr} (Version 1.1.1; Wickham, Hester, \&
Francois, 2017), \emph{RefManageR} (Version 1.2.0; McLean, 2018), and
\emph{xaringan} (Version 0.6.7; Xie, 2018b) for all our analyses.

\hypertarget{results}{%
\section{Results}\label{results}}

\includegraphics{report_files/figure-latex/barplots-1.pdf}

\hypertarget{discussion}{%
\section{Discussion}\label{discussion}}

\newpage

\hypertarget{references}{%
\section{References}\label{references}}

\begingroup
\setlength{\parindent}{-0.5in}
\setlength{\leftskip}{0.5in}

\hypertarget{refs}{}
\leavevmode\hypertarget{ref-R-ggthemes}{}%
Arnold, J. B. (2018). \emph{Ggthemes: Extra themes, scales and geoms for
'ggplot2'}. Retrieved from
\url{https://CRAN.R-project.org/package=ggthemes}

\leavevmode\hypertarget{ref-R-papaja}{}%
Aust, F., \& Barth, M. (2018). \emph{Papaja: Prepare reproducible apa
journal articles with r markdown}. Retrieved from
\url{https://github.com/crsh/papaja}

\leavevmode\hypertarget{ref-R-RefManageR}{}%
McLean, M. W. (2018). \emph{RefManageR: Straightforward 'bibtex' and
'biblatex' bibliography management}. Retrieved from
\url{https://CRAN.R-project.org/package=RefManageR}

\leavevmode\hypertarget{ref-R-here}{}%
Müller, K. (2017). \emph{Here: A simpler way to find your files}.
Retrieved from \url{https://CRAN.R-project.org/package=here}

\leavevmode\hypertarget{ref-R-patchwork}{}%
Pedersen, T. L. (2017). \emph{Patchwork: The composer of ggplots}.
Retrieved from \url{https://github.com/thomasp85/patchwork}

\leavevmode\hypertarget{ref-R-base}{}%
R Core Team. (2018). \emph{R: A language and environment for statistical
computing}. Vienna, Austria: R Foundation for Statistical Computing.
Retrieved from \url{https://www.R-project.org/}

\leavevmode\hypertarget{ref-R-ggalt}{}%
Rudis, B., Bolker, B., \& Schulz, J. (2017). \emph{Ggalt: Extra
coordinate systems, 'geoms', statistical transformations, scales and
fonts for 'ggplot2'}. Retrieved from
\url{https://CRAN.R-project.org/package=ggalt}

\leavevmode\hypertarget{ref-R-ggplot2}{}%
Wickham, H., Chang, W., Henry, L., Pedersen, T. L., Takahashi, K.,
Wilke, C., \& Woo, K. (2018). \emph{Ggplot2: Create elegant data
visualisations using the grammar of graphics}.

\leavevmode\hypertarget{ref-R-dplyr}{}%
Wickham, H., François, R., Henry, L., \& Müller, K. (2018). \emph{Dplyr:
A grammar of data manipulation}. Retrieved from
\url{https://CRAN.R-project.org/package=dplyr}

\leavevmode\hypertarget{ref-R-readr}{}%
Wickham, H., Hester, J., \& Francois, R. (2017). \emph{Readr: Read
rectangular text data}. Retrieved from
\url{https://CRAN.R-project.org/package=readr}

\leavevmode\hypertarget{ref-R-knitr}{}%
Xie, Y. (2018a). \emph{Knitr: A general-purpose package for dynamic
report generation in r}. Retrieved from
\url{https://CRAN.R-project.org/package=knitr}

\leavevmode\hypertarget{ref-R-xaringan}{}%
Xie, Y. (2018b). \emph{Xaringan: Presentation ninja}. Retrieved from
\url{https://github.com/yihui/xaringan}

\endgroup


\clearpage
\renewcommand{\listtablename}{Table captions}
\listoftables

\clearpage
\renewcommand{\listfigurename}{Figure captions}
\listoffigures

\theendnotes


\end{document}
